%% Based on a TeXnicCenter-Template by Gyorgy SZEIDL.
%%%%%%%%%%%%%%%%%%%%%%%%%%%%%%%%%%%%%%%%%%%%%%%%%%%%%%%%%%%%%

%------------------------------------------------------------
%
\documentclass[pdftex, 12pt]{report}%
%Options -- Point size:  10pt (default), 11pt, 12pt
%        -- Paper size:  letterpaper (default), a4paper, a5paper, b5paper
%                        legalpaper, executivepaper
%        -- Orientation  (portrait is the default)
%                        landscape
%        -- Print size:  oneside (default), twoside
%        -- Quality      final(default), draft
%        -- Title page   notitlepage, titlepage(default)
%        -- Columns      onecolumn(default), twocolumn
%        -- Equation numbering (equation numbers on the right is the default)
%                        leqno
%        -- Displayed equations (centered is the default)
%                        fleqn (equations start at the same distance from the right side)
%        -- Open bibliography style (closed is the default)
%                        openbib
% For instance the command
%           \documentclass[a4paper,12pt,leqno]{article}
% ensures that the paper size is a4, the fonts are typeset at the size 12p
% and the equation numbers are on the left side
%
\usepackage{amsmath}%
\usepackage{amsfonts}%
\usepackage{amssymb}%
\usepackage{graphicx}
\usepackage[margin=1in]{geometry}
\usepackage[onehalfspacing]{setspace}
\usepackage{titlesec}
\usepackage{fancyhdr}
\setlength{\headheight}{25pt}
\pagestyle{fancy}
\usepackage{enumerate}
\usepackage[pagebackref]{hyperref}
\hypersetup{pdfstartview=FitH}
\usepackage{bookmark}
\bookmarksetup{numbered, open}

%-------------------------------------------
\newtheorem{theorem}{Theorem}
\newtheorem{acknowledgement}[theorem]{Acknowledgement}
\newtheorem{algorithm}[theorem]{Algorithm}
\newtheorem{axiom}[theorem]{Axiom}
\newtheorem{case}[theorem]{Case}
\newtheorem{claim}[theorem]{Claim}
\newtheorem{conclusion}[theorem]{Conclusion}
\newtheorem{condition}[theorem]{Condition}
\newtheorem{conjecture}[theorem]{Conjecture}
\newtheorem{corollary}[theorem]{Corollary}
\newtheorem{criterion}[theorem]{Criterion}
\newtheorem{definition}[theorem]{Definition}
\newtheorem{example}[theorem]{Example}
\newtheorem{exercise}[theorem]{Exercise}
\newtheorem{lemma}[theorem]{Lemma}
\newtheorem{notation}[theorem]{Notation}
\newtheorem{problem}[theorem]{Problem}
\newtheorem{proposition}[theorem]{Proposition}
\newtheorem{remark}[theorem]{Remark}
\newtheorem{solution}[theorem]{Solution}
\newtheorem{summary}[theorem]{Summary}
\newenvironment{proof}[1][Proof]{\textbf{#1.} }{\ \rule{0.5em}{0.5em}}

%-------------------------------------------

\renewcommand{\thesubsection}{\arabic{subsection}}


%-------------------------------------------


\begin{document}


\title{\Huge\bfseries Smoothie \vspace{20pt}\\* Solutions to Riemannian Geometry, do Carmo}

\author{Kevin Ou\thanks{This is for making an acknowledgement.}
	\vspace{10pt}\\Department of Mathematical Sciences
	\\Carnegie Mellon University}
\date{\today}
\maketitle


\begin{abstract}
This is a joint project by the graduate students taking 21-759 Differential Geometry offered in Spring 2016 at Carnegie Mellon University. Participants type up solutions to the exercises at the end of each chapter of the text book Riemannian Geometry by Manfredo Perdigao do Carmo, and supplement or elaborate the materials in the main body, hoping to better understand the subject matter, to create a document for future references, and to learn document collaboration with Git.
\end{abstract}

\clearpage


\lhead{} \chead{} \rhead{}


\setcounter{page}{1}
\pagenumbering{roman}

\setcounter{tocdepth}{1} % hide all subsections in TOC
\tableofcontents\chead{Contents}

\clearpage


\pagenumbering{Roman}
\setcounter{page}{1}

\chapter*{PREFACE}\chead{PREFACE}
\addcontentsline{toc}{chapter}{PREFACE}
% We will fill it in later for acknowledgement or something when we are almost done


\clearpage

\setcounter{chapter}{-1}

\pagenumbering{arabic}
\setcounter{page}{1}


\chapter{DIFFERENTIABLE MANIFOLDS}
\chead{DIFFERENTIABLE MANIFOLDS}

\section*{NOTES}
\addcontentsline{toc}{section}{NOTES}

\section*{EXERCISES}
\addcontentsline{toc}{section}{EXERCISES}
	% Kevin Ou

{
% add braces so that commands defined for convenience are restricted locally to this file only

\subsection{Product manifold and quotient manifold by group action}

% coordinate map for convenience
\newcommand*{\x}{\mathbf{x}}
\newcommand*{\y}{\mathbf{y}}
\newcommand*{\z}{\mathbf{z}}
% isomorphism (category as input)
\newcommand*{\iso}[1]{\stackrel{\mathrm{#1.}}{\cong}}

\begin{proof}
\begin{enumerate}[(a)]
	\item
	Clearly, $M \times N$ is Hausdorff and second countable, and $\{U_\alpha \times V_\beta\}$ covers $M \times N$, with $\z_{\alpha,\beta}=(\x_\alpha,\y_\beta)$ being homeomorphic.
	\par
	Next, we shall show that the atlas is differentiable. Fix $(U_{\alpha_i} \times V_{\beta_i}, \z_{\alpha_i,\beta_i}) =: (W_i,\phi_i)$, where $i=1,2$, such that $O := \cap_{i=1}^2 \phi_i(W_i) \ne \emptyset$. Then the map
	\[\phi_2^{-1} \circ \phi_1: \phi_1^{-1}(O) \ni (p,q) \mapsto \left(\x_{\alpha_2}^{-1} \circ \x_{\alpha_1}(p), \y_{\beta_2}^{-1} \circ \y_{\beta_1}(q)\right) \in \phi_2^{-1}(O), p\in\mathbb{R}^{\mathrm{dim }M}, q\in\mathbb{R}^{\mathrm{dim }N}\]
	is clearly differentiable, by definition of $\{(U_\alpha,\x_\alpha)\}$ and $\{(V_\beta,\y_\beta)\}$ being differentiable structures on $M$ and $N$, respectively. Hence by definition, $\{U_\alpha \times V_\beta\}$ is a differentiable atlas on $M \times N$, and thus induces a unique differentiable structure on $M \times N$ (note that $\{(U_\alpha \times V_\beta,\z_{\alpha,\beta})\}$ is not maximal). So $M \times N$ equipped with the given atlas above is a differentiable manifold.
	\par
	Finally, we shall show that the projection onto $M$, 
	\[\pi_1: M \times N \ni (x,y) \mapsto x \in M, x \in M, y \in N,\]
	is differentiable. Fix $(x,y) \in M \times N$, and take, for convenience, $(U_\alpha, \x_\alpha)$ and $(V_\beta, \y_\beta)$ as the charts, with $p:= \x_\alpha^{-1}(x) \in U_\alpha$ and $q:= \y_\beta^{-1}(y) \in V_\beta$. Then the local expression of $\pi_1$ in the chosen charts is 
	\[ \left[x_\alpha^{-1} \circ \pi_1 \circ (x_\alpha, y_\beta)\right](p,q) = x_\alpha^{-1}[\pi_1(x,y)] = x_\alpha^{-1}(x) = p,\]
	which is obviously differentiable. Thus $\pi_1 \in C^\infty(M \times N; M)$.	By symmetry, the projection onto $N$, $\pi_2$, is also differentiable.
	
	\item
	We shall prove a slightly more general result: $\prod_{n=1}^N (M_n/G_n) \iso{diff} \left(\prod_{n=1}^N M_n\right)/\left(\prod_{n=1}^N G_n\right)$, where each group $G_n$ is a properly discontinuous action on the differentiable manifold $M_n$. In view of Example 4.8 and Part (a), $\prod_{n=1}^N (M_n/G_n)$ and $M:= \prod_{n=1}^N M_n$ are differentiable manifolds with the associated product differentiable structures. We shall show that $G:= \prod_{n=1}^N G_n$ acts on $M$ properly discontinuously and that the associated quotient differentiable structure coincides with the product differentiable structure of $\prod_{n=1}^N (M_n/G_n)$.
	\par
	As a corollary, $\left(S^1\right)^n \iso{diff} T^n$ follows from the fact that $S^1 \iso{diff} \mathbb{R}/\mathbb{Z} \iso{diff} T^1$.
	\par
	We first check that $G$ is a properly discontinuous group action on $M$. Fix $g_n \in G_n$ for every $n$, since $M_n \ni p_n \mapsto g_n p_n \in M_n$ is a diffeomorphic automorphism on $M_n$,
	\[ M \ni p:= (p_1,\dots,p_N) \mapsto (g_1 p_1, \dots, g_N p_N) =:g p \in M, p_n \in M_n \]
	is also a diffeomorphic automorphism on $M$. Now fix $p_n \in M_n$, by proper discontinuity of $G_n$ on $M_n$, we can find an open subset of $M_n$, $U_n \in p_n$, so as ${U_n \cap g_n(U_n) = \emptyset}, { \forall g_n \in G_n\backslash\{e_n\} }$. Note that $V:=\prod_{n=1}^{N} U_n$ is and open neighborhood of $p \in M$, and that
	\[V \cap g(V) = \left[\prod_{n=1}^N U_n \right] \cap \left[\prod_{n=1}^N g_n(U_n) \right] = \prod_{n=1}^N \left[U_n \cap g_n(U_n)\right] = \emptyset, \forall g \in G\backslash\{e\}.\]
	So $G$ is a properly discontinuous action on $M$.
	\par
	Now we shall show that the product differentiable structure is the same as the quotient differentiable structure, which means finding a diffeomorphism $\phi: M/G \rightarrow \prod_{n=1}^N (M_n/G_n)$, and this completes the proof. For each $n$, let $\{(O_n^{\alpha_n}, \x_n^{\alpha_n})\}$ be a differentiable atlas on $M_n$ with	$ \x_n^{\alpha_n}(O_n^{\alpha_n}) \cap g_n\left(\x_n^{\alpha_n}(O_n^{\alpha_n})\right) = \emptyset, \forall g_n \in G_n\backslash\{e_n\}, \alpha^n$, and ${ q_n: M_n \ni p_n \mapsto [p_n]_{G_n} \in M_n/G_n }$ be the quotient map (projection). Then we know from the construction of quotient manifold by group action, $\{(O_n^{\alpha_n}, \y_n^{\alpha_n}:= q_n \circ \x_n^{\alpha_n})\}$ induces the quotient differentiable structure on $M_n/G_n$. By Part (a), $\{(O^\alpha:= \prod_{n=1}^N O_n^{\alpha_n}, \y^\alpha:= (\y_1^{\alpha_1}, \dots, \y_N^{\alpha_N})\}$ gives a differentiable atlas on $\prod_{n=1}^N (M_n/G_n)$. Similarly, ${ \{(O^{\alpha}, \x^{\alpha}:= (\x_1^{\alpha_1}, \dots, \x_N^{\alpha_N})\} }$ is a differentiable atlas on $M$, and from the proof of proper discontinuity, ${ \x^\alpha(O^\alpha) \cap g\left(\x^\alpha(O^\alpha)\right) = \emptyset }, { \forall g \in G\backslash\{e\} }$. Hence, ${ \{O^\alpha, \z^\alpha:= q \circ \x^\alpha\} }$ determines the quotient differentiable structure on $M/G$, where ${ q:= (q_1, \dots, q_N) }$. Consider ${ \phi:M/G \ni [p]_G \mapsto ([p_1]_{G_1}, \dots, [p_N]_{G_N}) \in \prod_{n=1}^N M_n/G_n }$, which is obviously well-defined and bijective. For $[p]_G \in M/G$, let ${ (O^\alpha, \z^\alpha) }$ be such that ${ O^\alpha \ni z:=(\z^\alpha)^{-1}([p]_G) }$, Observe that 
	\[ \left( \y^\alpha \right)^{-1} \circ \phi \circ \z^\alpha (z) 
	= \left( \y^\alpha \right)^{-1} \left[ \phi \left([p]_G\right) \right]
	= \left( \y^\alpha \right)^{-1}\left(([p_1]_{G_1}, \dots, [p_N]_{G_N})\right)
	= (z_1, \dots, z_N) = z
	\]
	and hence ${ \phi \in C^\infty(M/G;\prod_{n=1}^N (M_n/G_n)) }$ as claimed. Thus, ${ \prod_{n=1}^N (M_n/G_n) \iso{diff} M/G }$.
	
\end{enumerate}
\end{proof}

} % Kevin
	% Kevin Ou

{
% add braces so that commands defined for convenience are restricted locally to this file only

\newcommand*{\R}{\mathbb{R}} % real numbers
\renewcommand*{\d}{\mathrm{d}\,} % differential, exterior derivative
\newcommand*{\D}{\mathrm{D}\,} % Jacobian, covariant derivative


\subsection{Orientability of tangent bundle}

\begin{proof}
To show that $TM$ is orientable, we have to check that the determinant of every transition map in a certain atlas is always positive.
\par
Let $\{(U,\varphi)\}$ be a chart on $M$, then define $\tilde{\varphi}:U\times \R^n \rightarrow TM$ via $\tilde{\varphi}(x,a):=\left(\varphi(x), a^i \frac{\partial}{\partial x_i} \right)$, with $x,a\in \R^n, x\in U$. Clearly $\left(U\times \R^n, \tilde{\varphi} \right)$ is a chart on $TM$.
\par
Fix two charts $\left(U\times \R^n, \tilde{\varphi} \right)$ and $\left(V\times \R^n, \tilde{\psi} \right)$ on $TM$ with $\varphi(U) \cap \psi(V) \ne \emptyset$. Let ${x\in U}, {y\in V}, a,b \in \R^n$ be such that $\tilde{\varphi}(x,a)=\left(\varphi(x), a^i \frac{\partial}{\partial x_i} \right) = \left(\psi(y), b^i \frac{\partial}{\partial y_i} \right) = \tilde{\psi}(y,b)$. Thus, $x = \varphi^{-1}\circ\psi(y)$.
\par
Note that $t\mapsto \varphi\left(\varphi^{-1}(p)+ta \right) = \varphi(x+ta)$ is a representative of $a^i \frac{\partial}{\partial x_i}$, where $\varphi(x)=:p \in M$. Thus,
$$a = \frac{\d}{\d t}\bigg|_{t=0}\varphi^{-1}\left(\varphi(x+ta)\right) = \frac{\d}{\d t}\bigg|_{t=0}\varphi^{-1}\left(\psi(y+tb)\right) = D\left(\varphi^{-1}\circ\psi \right)(y) b.$$
Hence, $\tilde{\varphi}^{-1}\circ\tilde{\psi}(y,b) = (x,a) = \left(\varphi^{-1}\circ\psi(y), \D\left(\varphi^{-1}\circ\psi\right)(y) b \right)$. Then,
$$\D\left(\tilde{\varphi}^{-1}\circ\tilde{\psi}\right)(y,b) = 
\begin{bmatrix}
	\D\left(\varphi^{-1}\circ\psi\right)(y) & 0 \\
	\cdots & \D\left(\varphi^{-1}\circ\psi\right)(y)
\end{bmatrix},
$$
which obviously has positive determinant $\det \D\left(\tilde{\varphi}^{-1}\circ\tilde{\psi}\right)(y,b) = \left[\det \D\left(\varphi^{-1}\circ\psi\right)(y) \right]^2 >0$.
\par
So by definition, $TM$ is orientable.


\end{proof}

} % Antoine
	%{
% add braces so that commands defined for convenience are restricted locally to this file only

\subsection{Orientability of tangent bundle}

\begin{proof}



\end{proof}

}
	%{
% add braces so that commands defined for convenience are restricted locally to this file only

\subsection{Orientability of tangent bundle}

\begin{proof}



\end{proof}

}
	%{
% add braces so that commands defined for convenience are restricted locally to this file only

\subsection{Orientability of tangent bundle}

\begin{proof}



\end{proof}

}


\chapter{RIEMANNIAN METRICS}
\chead{RIEMANNIAN METRICS}

\section*{NOTES}
\addcontentsline{toc}{section}{NOTES}

\section*{EXERCISES}
\addcontentsline{toc}{section}{EXERCISES}
	%{
% add braces so that commands defined for convenience are restricted locally to this file only

\subsection{Orientability of tangent bundle}

\begin{proof}



\end{proof}

}
	%{
% add braces so that commands defined for convenience are restricted locally to this file only

\subsection{Orientability of tangent bundle}

\begin{proof}



\end{proof}

}
	%{
% add braces so that commands defined for convenience are restricted locally to this file only

\subsection{Orientability of tangent bundle}

\begin{proof}



\end{proof}

}


\chapter{AFFINE CONNECTIONS; RIEMANNIAN CONNECTIONS}
\chead{AFFINE CONNECTIONS; RIEMANNIAN CONNECTIONS}

\section*{NOTES}
\addcontentsline{toc}{section}{NOTES}

\section*{EXERCISES}
\addcontentsline{toc}{section}{EXERCISES}
	%{
% add braces so that commands defined for convenience are restricted locally to this file only

\subsection{Orientability of tangent bundle}

\begin{proof}



\end{proof}

}
	%{
% add braces so that commands defined for convenience are restricted locally to this file only

\subsection{Orientability of tangent bundle}

\begin{proof}



\end{proof}

}
	%{
% add braces so that commands defined for convenience are restricted locally to this file only

\subsection{Orientability of tangent bundle}

\begin{proof}



\end{proof}

}


\chapter{GEODESICS; CONVEX NEIGHBORHOODS}
\chead{GEODESICS; CONVEX NEIGHBORHOODS}

\section*{NOTES}
\addcontentsline{toc}{section}{NOTES}

\section*{EXERCISES}
\addcontentsline{toc}{section}{EXERCISES}
	%{
% add braces so that commands defined for convenience are restricted locally to this file only

\subsection{Orientability of tangent bundle}

\begin{proof}



\end{proof}

}
	%{
% add braces so that commands defined for convenience are restricted locally to this file only

\subsection{Orientability of tangent bundle}

\begin{proof}



\end{proof}

}
	%{
% add braces so that commands defined for convenience are restricted locally to this file only

\subsection{Orientability of tangent bundle}

\begin{proof}



\end{proof}

}


\chapter{CURVATURE}
\chead{CURVATURE}

\section*{NOTES}
\addcontentsline{toc}{section}{NOTES}

\section*{EXERCISES}
\addcontentsline{toc}{section}{EXERCISES}
	%{
% add braces so that commands defined for convenience are restricted locally to this file only

\subsection{Orientability of tangent bundle}

\begin{proof}



\end{proof}

}
	%{
% add braces so that commands defined for convenience are restricted locally to this file only

\subsection{Orientability of tangent bundle}

\begin{proof}



\end{proof}

}
	%{
% add braces so that commands defined for convenience are restricted locally to this file only

\subsection{Orientability of tangent bundle}

\begin{proof}



\end{proof}

}


\chapter{JACOBI FIELDS}
\chead{JACOBI FIELDS}

\section*{NOTES}
\addcontentsline{toc}{section}{NOTES}

\section*{EXERCISES}
\addcontentsline{toc}{section}{EXERCISES}
	%{
% add braces so that commands defined for convenience are restricted locally to this file only

\subsection{Orientability of tangent bundle}

\begin{proof}



\end{proof}

}
	%{
% add braces so that commands defined for convenience are restricted locally to this file only

\subsection{Orientability of tangent bundle}

\begin{proof}



\end{proof}

}
	%{
% add braces so that commands defined for convenience are restricted locally to this file only

\subsection{Orientability of tangent bundle}

\begin{proof}



\end{proof}

}


\chapter{ISOMETRIC IMMERSIONS}
\chead{ISOMETRIC IMMERSIONS}

\section*{NOTES}
\addcontentsline{toc}{section}{NOTES}

\section*{EXERCISES}
\addcontentsline{toc}{section}{EXERCISES}
	%{
% add braces so that commands defined for convenience are restricted locally to this file only

\subsection{Orientability of tangent bundle}

\begin{proof}



\end{proof}

}
	%{
% add braces so that commands defined for convenience are restricted locally to this file only

\subsection{Orientability of tangent bundle}

\begin{proof}



\end{proof}

}
	%{
% add braces so that commands defined for convenience are restricted locally to this file only

\subsection{Orientability of tangent bundle}

\begin{proof}



\end{proof}

}


\chapter{COMPLETE MANIFOLDS; HOPF-RINOW AND HADAMARD THEOREMS}
\chead{COMPLETE MANIFOLDS; HOPF-RINOW AND HADAMARD THEOREMS}

\section*{NOTES}
\addcontentsline{toc}{section}{NOTES}

\section*{EXERCISES}
\addcontentsline{toc}{section}{EXERCISES}
	%{
% add braces so that commands defined for convenience are restricted locally to this file only

\subsection{Orientability of tangent bundle}

\begin{proof}



\end{proof}

}
	%{
% add braces so that commands defined for convenience are restricted locally to this file only

\subsection{Orientability of tangent bundle}

\begin{proof}



\end{proof}

}
	%{
% add braces so that commands defined for convenience are restricted locally to this file only

\subsection{Orientability of tangent bundle}

\begin{proof}



\end{proof}

}


\chapter{SPACES OF CONSTANT CURVATURE}
\chead{SPACES OF CONSTANT CURVATURE}

\section*{NOTES}
\addcontentsline{toc}{section}{NOTES}

\section*{EXERCISES}
\addcontentsline{toc}{section}{EXERCISES}
	%{
% add braces so that commands defined for convenience are restricted locally to this file only

\subsection{Orientability of tangent bundle}

\begin{proof}



\end{proof}

}
	%{
% add braces so that commands defined for convenience are restricted locally to this file only

\subsection{Orientability of tangent bundle}

\begin{proof}



\end{proof}

}
	%{
% add braces so that commands defined for convenience are restricted locally to this file only

\subsection{Orientability of tangent bundle}

\begin{proof}



\end{proof}

}


\chapter{VARIATIONS OF ENERGY}
\chead{VARIATIONS OF ENERGY}

\section*{NOTES}
\addcontentsline{toc}{section}{NOTES}

\section*{EXERCISES}
\addcontentsline{toc}{section}{EXERCISES}
	%{
% add braces so that commands defined for convenience are restricted locally to this file only

\subsection{Orientability of tangent bundle}

\begin{proof}



\end{proof}

}
	%{
% add braces so that commands defined for convenience are restricted locally to this file only

\subsection{Orientability of tangent bundle}

\begin{proof}



\end{proof}

}
	%{
% add braces so that commands defined for convenience are restricted locally to this file only

\subsection{Orientability of tangent bundle}

\begin{proof}



\end{proof}

}


\chapter{THE RAUCH COMPARISON THEOREM}
\chead{THE RAUCH COMPARISON THEOREM}

\section*{NOTES}
\addcontentsline{toc}{section}{NOTES}

\section*{EXERCISES}
\addcontentsline{toc}{section}{EXERCISES}
	%{
% add braces so that commands defined for convenience are restricted locally to this file only

\subsection{Orientability of tangent bundle}

\begin{proof}



\end{proof}

}
	%{
% add braces so that commands defined for convenience are restricted locally to this file only

\subsection{Orientability of tangent bundle}

\begin{proof}



\end{proof}

}
	%{
% add braces so that commands defined for convenience are restricted locally to this file only

\subsection{Orientability of tangent bundle}

\begin{proof}



\end{proof}

}


\chapter{THE MORSE INDEX THEOREM}
\chead{THE MORSE INDEX THEOREM}

\section*{NOTES}
\addcontentsline{toc}{section}{NOTES}

\section*{EXERCISES}
\addcontentsline{toc}{section}{EXERCISES}
	%{
% add braces so that commands defined for convenience are restricted locally to this file only

\subsection{Orientability of tangent bundle}

\begin{proof}



\end{proof}

}
	%{
% add braces so that commands defined for convenience are restricted locally to this file only

\subsection{Orientability of tangent bundle}

\begin{proof}



\end{proof}

}
	%{
% add braces so that commands defined for convenience are restricted locally to this file only

\subsection{Orientability of tangent bundle}

\begin{proof}



\end{proof}

}


\chapter{THE FUNDAMENTAL GROUP OF MANIFOLDS OF NEGATIVE CURVATURE}
\chead{THE FUNDAMENTAL GROUP OF MANIFOLDS OF NEGATIVE CURVATURE}

\section*{NOTES}
\addcontentsline{toc}{section}{NOTES}

\section*{EXERCISES}
\addcontentsline{toc}{section}{EXERCISES}
	%{
% add braces so that commands defined for convenience are restricted locally to this file only

\subsection{Orientability of tangent bundle}

\begin{proof}



\end{proof}

}
	%{
% add braces so that commands defined for convenience are restricted locally to this file only

\subsection{Orientability of tangent bundle}

\begin{proof}



\end{proof}

}
	%{
% add braces so that commands defined for convenience are restricted locally to this file only

\subsection{Orientability of tangent bundle}

\begin{proof}



\end{proof}

}


\chapter{THE SPHERE THEOREM}
\chead{THE SPHERE THEOREM}

\section*{NOTES}
\addcontentsline{toc}{section}{NOTES}

\section*{EXERCISES}
\addcontentsline{toc}{section}{EXERCISES}
	%{
% add braces so that commands defined for convenience are restricted locally to this file only

\subsection{Orientability of tangent bundle}

\begin{proof}



\end{proof}

}
	%{
% add braces so that commands defined for convenience are restricted locally to this file only

\subsection{Orientability of tangent bundle}

\begin{proof}



\end{proof}

}
	%{
% add braces so that commands defined for convenience are restricted locally to this file only

\subsection{Orientability of tangent bundle}

\begin{proof}



\end{proof}

}



% Let's keep the following illustration for now
%------------------------------------------------------------
\chapter{ILLUSTRATION}
\chead{ILLUSTRATION}

\section{Introduction}

\noindent The front matter has various entries such as\\
\hspace*{\fill}\verb" \title", \verb"\author", \verb"\date", and
\verb"\thanks"\hspace*{\fill}\\
You should replace their arguments with your own.

This text is the body of your article. You may delete everything between the commands\\
\hspace*{\fill} \verb"\begin{document}" \ldots \verb"\end{document}"
\hspace*{\fill}\\in this file to start with a blank document.


\section{The Most Important Features}

\noindent Sectioning commands. The first one is the\\
\hspace*{\fill} \verb"\section{The Most Important Features}" \hspace*{\fill}\\
command. Below you shall find examples for further sectioning commands:

\subsection{Subsection}
Subsection text.

\subsubsection{Subsubsection}
Subsubsection text.

\paragraph{Paragraph}
Paragraph text.

\subparagraph{Subparagraph}Subparagraph text.\vspace{2mm}

Select a part of the text then click on the button Emphasize (H!), or Bold (Fs), or
Italic (Kt), or Slanted (Kt) to typeset \emph{Emphasize}, \textbf{Bold},
\textit{Italics}, \textsl{Slanted} texts.

You can also typeset \textrm{Roman}, \textsf{Sans Serif}, \textsc{Small Caps}, and
\texttt{Typewriter} texts.

You can also apply the special, mathematics only commands $\mathbb{BLACKBOARD}$
$\mathbb{BOLD}$, $\mathcal{CALLIGRAPHIC}$, and $\mathfrak{fraktur}$. Note that
blackboard bold and calligraphic are correct only when applied to uppercase letters A
through Z.

You can apply the size tags -- Format menu, Font size submenu -- {\tiny tiny},
{\scriptsize scriptsize}, {\footnotesize footnotesize}, {\small small}, {\normalsize
normalsize}, {\large large}, {\Large Large}, {\LARGE LARGE}, {\huge huge} and {\Huge
Huge}.

You can use the \verb"\begin{quote} etc. \end{quote}" environment for typesetting
short quotations. Select the text then click on Insert, Quotations, Short Quotations:

\begin{quote}
The buck stops here. \emph{Harry Truman}

Ask not what your country can do for you; ask what you can do for your
country. \emph{John F Kennedy}

I am not a crook. \emph{Richard Nixon}

I did not have sexual relations with that woman, Miss Lewinsky. \emph{Bill Clinton}
\end{quote}

The Quotation environment is used for quotations of more than one paragraph. Following
is the beginning of \emph{The Jungle Books} by Rudyard Kipling. (You should select
the text first then click on Insert, Quotations, Quotation):

\begin{quotation}
It was seven o'clock of a very warm evening in the Seeonee Hills when Father Wolf woke
up from his day's rest, scratched himself, yawned  and spread out his paws one after
the other to get rid of sleepy feeling in their tips. Mother Wolf lay with her big gray
nose dropped across her four tumbling, squealing cubs, and the moon shone into the
mouth of the cave where they all lived. ``\emph{Augrh}'' said Father Wolf, ``it is time
to hunt again.'' And he was going to spring down hill when a little shadow with a bushy
tail crossed the threshold and whined: ``Good luck go with you, O Chief of the Wolves;
and good luck and strong white teeth go with the noble children, that they may never
forget the hungry in this world.''

It was the jackal---Tabaqui the Dish-licker---and the wolves of India despise Tabaqui
because he runs about making mischief, and telling tales, and eating rags and pieces of
leather from the village rubbish-heaps. But they are afraid of him too, because
Tabaqui, more than any one else in the jungle, is apt to go mad, and then he forgets
that he was afraid of anyone, and runs through the forest biting everything in his way.
\end{quotation}

Use the Verbatim environment if you want \LaTeX\ to preserve spacing, perhaps when
including a fragment from a program such as:
\begin{verbatim}
#include <iostream>         // < > is used for standard libraries.
void main(void)             // ''main'' method always called first.
{
 cout << ''This is a message.'';
                            // Send to output stream.
}
\end{verbatim}
(After selecting the text click on Insert, Code Environments, Code.)


\subsection{Mathematics and Text}

It holds \cite{KarelRektorys} the following
\begin{theorem}
(The Currant minimax principle.) Let $T$ be completely continuous selfadjoint operator
in a Hilbert space $H$. Let $n$ be an arbitrary integer and let $u_1,\ldots,u_{n-1}$ be
an arbitrary system of $n-1$ linearly independent elements of $H$. Denote
\begin{equation}
\max_{\substack{v\in H, v\neq
0\\(v,u_1)=0,\ldots,(v,u_n)=0}}\frac{(Tv,v)}{(v,v)}=m(u_1,\ldots, u_{n-1})
\label{eqn10}
\end{equation}
Then the $n$-th eigenvalue of $T$ is equal to the minimum of these maxima, when
minimizing over all linearly independent systems $u_1,\ldots u_{n-1}$ in $H$,
\begin{equation}
\mu_n = \min_{\substack{u_1,\ldots, u_{n-1}\in H}} m(u_1,\ldots, u_{n-1}) \label{eqn20}
\end{equation}
\end{theorem}
The above equations are automatically numbered as equation (\ref{eqn10}) and
(\ref{eqn20}).

\subsection{List Environments}

You can create numbered, bulleted, and description lists using the tag popup
at the bottom left of the screen.

\begin{enumerate}
\item List item 1

\item List item 2

\begin{enumerate}
\item A list item under a list item.

The typeset style for this level is different than the screen style. \ The
screen shows a lower case alphabetic character followed by a period while the
typeset style uses a lower case alphabetic character surrounded by parentheses.

\item Just another list item under a list item.

\begin{enumerate}
\item Third level list item under a list item.

\begin{enumerate}
\item Fourth and final level of list items allowed.
\end{enumerate}
\end{enumerate}
\end{enumerate}
\end{enumerate}

\begin{itemize}
\item Bullet item 1

\item Bullet item 2

\begin{itemize}
\item Second level bullet item.

\begin{itemize}
\item Third level bullet item.

\begin{itemize}
\item Fourth (and final) level bullet item.
\end{itemize}
\end{itemize}
\end{itemize}
\end{itemize}

\begin{description}
\item[Description List] Each description list item has a term followed by the
description of that term. Double click the term box to enter the term, or to
change it.

\item[Bunyip] Mythical beast of Australian Aboriginal legends.
\end{description}

\subsection{Theorem-like Environments}

The following theorem-like environments (in alphabetical order) are available
in this style.

\begin{acknowledgement}
This is an acknowledgement
\end{acknowledgement}

\begin{algorithm}
This is an algorithm
\end{algorithm}

\begin{axiom}
This is an axiom
\end{axiom}

\begin{case}
This is a case
\end{case}

\begin{claim}
This is a claim
\end{claim}

\begin{conclusion}
This is a conclusion
\end{conclusion}

\begin{condition}
This is a condition
\end{condition}

\begin{conjecture}
This is a conjecture
\end{conjecture}

\begin{corollary}
This is a corollary
\end{corollary}

\begin{criterion}
This is a criterion
\end{criterion}

\begin{definition}
This is a definition
\end{definition}

\begin{example}
This is an example
\end{example}

\begin{exercise}
This is an exercise
\end{exercise}

\begin{lemma}
This is a lemma
\end{lemma}

\begin{proof}
This is the proof of the lemma.
\end{proof}

\begin{notation}
This is notation
\end{notation}

\begin{problem}
This is a problem
\end{problem}

\begin{proposition}
This is a proposition
\end{proposition}

\begin{remark}
This is a remark
\end{remark}

\begin{solution}
This is a solution
\end{solution}

\begin{summary}
This is a summary
\end{summary}

\begin{theorem}
This is a theorem
\end{theorem}

\begin{proof}
[Proof of the Main Theorem]This is the proof.
\end{proof}
\medskip

This text is a sample for a short bibliography. You can cite a book by making use of
the command \verb"\cite{KarelRektorys}": \cite{KarelRektorys}. Papers can be cited
similarly: \cite{Bertoti97}. If you want multiple citations to appear in a single set
of square brackets you must type all of the citation keys inside a single citation,
separating each with a comma. Here is an example: \cite{Bertoti97, Szeidl2001,
Carlson67}.

\begin{thebibliography}{9}                                                                                                %
\bibitem {KarelRektorys}Rektorys, K., \textit{Variational methods in Mathematics,
Science and Engineering}, D. Reidel Publishing Company,
Dordrecht-Hollanf/Boston-U.S.A., 2th edition, 1975

\bibitem {Bertoti97} \textsc{Bert\'{o}ti, E.}:\ \textit{On mixed variational formulation
of linear elasticity using nonsymmetric stresses and displacements}, International
Journal for Numerical Methods in Engineering., \textbf{42}, (1997), 561-578.

\bibitem {Szeidl2001} \textsc{Szeidl, G.}:\ \textit{Boundary integral equations for
plane problems in terms of stress functions of order one}, Journal of Computational and
Applied Mechanics, \textbf{2}(2), (2001), 237-261.

\bibitem {Carlson67}  \textsc{Carlson D. E.}:\ \textit{On G\"{u}nther's stress functions
for couple stresses}, Quart. Appl. Math., \textbf{25}, (1967), 139-146.
\end{thebibliography}


\appendix

\section{The First Appendix}

The appendix fragment is used only once. Subsequent appendices can be created
using the Section Section/Body Tag.
\end{document}
